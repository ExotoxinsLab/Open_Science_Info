\documentclass[11pt,a4paper]{article}
\usepackage[utf8]{inputenc}
\usepackage[T1]{fontenc}
\usepackage{amsmath}
\usepackage{amssymb}
\usepackage{graphicx}
\usepackage[english]{babel}
\usepackage[right=35mm, left=20mm]{geometry} %Allows me to change the width of the margins

\usepackage[colorlinks=true,linkcolor=blue,urlcolor=blue,citecolor=blue]{hyperref}
%for the hyperref to work, the document class has to have the option "final" and not "draft"

% To draw the directory graphics
\usepackage{forest}

\definecolor{folderbg}{RGB}{124,166,198}
\definecolor{folderborder}{RGB}{110,144,169}

\def\Size{4pt}
\tikzset{
	folder/.pic={
		\filldraw[draw=folderborder,top color=folderbg!50,bottom color=folderbg]
		(-1.05*\Size,0.2\Size+5pt) rectangle ++(.75*\Size,-0.2\Size-5pt);  
		\filldraw[draw=folderborder,top color=folderbg!50,bottom color=folderbg]
		(-1.15*\Size,-\Size) rectangle (1.15*\Size,\Size);
	}
}
% end of header for directory graphics

%For boxes highlight based on the importance
\usepackage{alertmessage}

% For captions in minipages
\usepackage{caption}

%for adding code chunks to latex (answer from Stock Exchange latex)
\usepackage{listings}
\usepackage{color}
\definecolor{dkgreen}{rgb}{0,0.6,0}
\definecolor{gray}{rgb}{0.5,0.5,0.5}
\definecolor{mauve}{rgb}{0.58,0,0.82}
\lstset{frame=tb,
	language=r,
	aboveskip=3mm,
	belowskip=3mm,
	showstringspaces=false,
	columns=flexible,
	basicstyle={\small\ttfamily},
	numbers=none,
	numberstyle=\tiny\color{gray},
	keywordstyle=\color{blue},
	commentstyle=\color{dkgreen},
	stringstyle=\color{mauve},
	breaklines=true,
	breakatwhitespace=true,
	tabsize=3
}

% To do check lists
\usepackage{enumitem} % it requires amssymb as well, see above
\newlist{todolist}{itemize}{2}
\newcommand{\checkbox}{\text{\fboxsep=-.15pt\fbox{\rule{0pt}{1.5ex}\rule{1.5ex}{0pt}}}}
\setlist[todolist]{label=\checkbox}


\usepackage{graphicx}
\graphicspath{{Figures/}} %Setting the graphicspath
\makeindex

%This is to modify the formatting of the index
\makeatletter
\renewcommand*\@idxitem{\par\hangindent 20\p@} %the indent size on lines after first line
\renewcommand*\subitem{\@idxitem \hspace*{20\p@}}
\renewcommand*\subsubitem{\@idxitem \hspace*{20\p@}}
\makeatother


\author{Your Name}
\title{Your Lab - Data Management Plan}
\date{Started on June 2022}

\usepackage{fancyhdr}
\fancyhf{}
\fancyhead[L]{\thepage}

\usepackage{tabto}

%Created with TeXstudio 4.2.2 (git n/a), Using Qt Version 5.9.5, compiled with Qt 5.9.5 R. LaTex version.
% pdfTeX 3.14159265-2.6-1.40.18 (TeX Live 2017/Debian)
% kpathsea version 6.2.3
% Copyright 2017 Han The Thanh (pdfTeX) et al.
% There is NO warranty.  Redistribution of this software is
% covered by the terms of both the pdfTeX copyright and
% the Lesser GNU General Public License.
% For more information about these matters, see the file
% named COPYING and the pdfTeX source.
% Primary author of pdfTeX: Han The Thanh (pdfTeX) et al.
% Compiled with libpng 1.6.34; using libpng 1.6.34
% Compiled with zlib 1.2.11; using zlib 1.2.11
% Compiled with poppler version 0.62.0


\begin{document}
	\maketitle
	
	%\thispagestyle{empty} %removes the page number below in the pages where the chapter number 
	%with page number appears (between chapter titles)
	
	
	%remove this section when in your final document... or not 
	\section{credits for this template}
	All sections and text explaining what should go in each section is copied directly from the CESSDA (https://www.cessda.eu/Training/DMEG) \href{URL}{PDF guide} called \textit{Adapt your Data Management Plan}
		

	\section[general info]{General Information}
	
	\NumTabs{8} % define 8 equally spaced tabs starting at the left margin
	% and spanning \linewidth
	\begin{itemize}
		\item\textbf{Title of the study}:\tab Your very important title.
		\item\textbf{Data steward}:\tab A name
		\item\textbf{Data of the plan}:\tab June 2022
	\end{itemize}

	\section{Description of the project}
	\subsection{Overview}
	What is the nature of the project?
	
	What is the research question?
	
	What is the project time line?
	\subsection{Principal researchers involved}
	Who are the main researchers involved?
	
	What are their contact details?
	\subsection{Collaborating researchers}
	What are their contact details and their roles in the project?
	
	\subsection{Funding organization}
	If funding is granted, what is the reference number of the funding granted?\\
	What is the project’s title in the funding contract?
	
	\subsection{Data producer}
	(I am still not sure what this means)
	Which organisation has the administrative responsibility for the data?
	
	\subsection{Project data contact}
	Who can be contacted about the project during and after it has finished?
	
	\subsection{Data owner(s)}
	Which organisation(s) own(s) the data?
	If several organisations are involved, which organisation owns what data?
	
	\subsection{Roles}
	Who is responsible for updating the DMP and making sure that it’s followed?
	Do project participants have any specific roles?
	What is the project time line?
	
	\subsection{Cost and resources}
	Are there costs you need to consider to buy specific software or hardware?
	Are there costs you need to consider for storage and backup?
	Are potential expenses and resources for (preparing the data for) archiving covered?
	What resources will be dedicated to data management ensuring that data will be FAIR?
	
	\section[data info]{Data description and collection or re-use of existing data}
	What kind of data will be used during the project?
	If you are reusing existing data: What is the scope, volume and format? How are
	different data sources integrated?
	If you are collecting new data can you clarify why this is necessary?
	
	\subsection{Origin of the data}
	
	\subsection{Types of data created}
	\subsection{Types of data re-used}
	
	\section{Organizing and Documenting the data}
	\subsection{Data collection}
	How will the data be collected?
	Is specific software or hardware or staff required?
	Who will be responsible for the data collection?
	During which period will the data be collected?
	Where will the data be collected?
	\subsection{Data organization}
	How will you organise your data?
	Will the data be organised in simple files or more complex databases?
	How will the data quality during the project be ensured?
	If data consists of many different file types (e.g. videos, text, photos), is it possible to structure the data in a logical way?
	\subsection{Data type and size}
	What type(s) of data will be collected?
	What is the scope, quantity and format of the material?
	After the project: What is the total amount of data collected (in MB/GB)?
	\subsection{Data format}
	In what format will your data be?
	Does the format change from the original to the processed/final data?
	Will your (final) data be available in an open format?
	\subsection{Directory structure and names}
	How will you structure and name your folders/directories?
	\subsection{File structure and names}
	How will you structure and name your folders?
	\subsection{Documentation}
	What documentation will be created during the different phases of the project?
	How will the documentation be structured?
	\subsection{Metadata}
	What metadata will be provided with the collected/ generated/ reused data?
	How will metadata for each object be created?
	Is there any program that can be used to document the data?
	Can metadata be added directly into the files or will the metadata be produced in another
	program or document?
	\subsection{Metadata standard (if applicable)}
	What metadata standard(s) will you use?
	
	\section{Processing your data}
	\subsection{Versioning}
	What is your strategy concerning versioning your data files (and scripts) during the project?
	Will you create and/or follow a convention for versioning your data?
	Who will be responsible for securing that a “Masterfile” will be maintained, documented
	and versioned according to the project guidelines?
	How can different versions of a data file be distinguished?
	
	\subsection{Interoperability}
	Will you make use of established software and hardware? If not, how does the software
	and hardware you use relate to other research?
	
	If applicable:
	Will you make use of established terminologies/ontologies (i.e. structured controlled
	vocabularies) in the project? If not, how do your terminologies relate to established
	ones?
	Which coding is used (if any)? Will you build on established coding schemes? If not,
	how does your coding relate to other research?
	
	\subsection{Data quality}
	How will data quality be evaluated?
	What data quality control measures will be used?
	
	\section{Storing data and metadata}
	\subsection{Storage}
	How and where will the (meta)data be stored during the project?
	For how long will the (meta)data be stored?
	
	\subsection{Backup}
	How, where and at what intervals will the (meta)data be backed-up?
	How will data be recovered in the case of a (meta)data loss incident?
	
	\subsection{Security}
	How will sensitive (meta)data be protected? (if applicable)
	How will (meta)data access be managed?
	
	\section{Protecting your data}
	\subsection{Ethical Review (if applicable)}
	Does your project require approval by a local ethics committee?
	How will possible ethical issues be taken into account, and codes of conduct followed?
	\subsection{Informed consent (if applicable)}
	Do you require informed consent for your project?
	If so, how will permission be obtained?
	How are consent files organised and stored?
	
	
	\subsection{(sensitive) Personal data /confidential information (if applicable)}
	How will access to (sensitive) personal data during the project be controlled?
	How will collaborators be granted access to the data in a secure way?
	If the research project is going to have data that includes confidential information or
	information that requires informed consent, is there a requirement to notify a privacy officer?
	Is there any confidential information within the material that requires special treatment
	and/or limits the access to it during/after the project?
	How will the material be protected during/after the project?
	How will permissions and restrictions be enforced?
	
	\subsection{Intellectual property rights (IPR) or Copyrights}
	Are there IPR or copyright issues to consider?
	Will permission be needed to collect/reuse the data?
	Will these rights be transferred to another organisation for data distribution and archiving?
	
	\subsection{Agreements (if applicable)}
	What are the agreements with other stakeholders?
	
	\subsection{Restrictions (if applicable)}
	Are there any other restrictions that need to be considered?
	
	\section{Archiving and publishing your data}
	\subsection{Archiving}
	How and where will the data be stored after the project’s completion?
	Will you archive your data in a trusted data repository?
	Will the application of a persistent identifier to your data be ensured?
	\subsection{Data formats}
	What formats will you provide your data in for archiving (and sharing)?
	Will specific software be required to process your data? Can this software be deposited
	with the data?
	\subsection{Access (if applicable)}
	Will your data be available (Open Access)?
	Will all data or only parts of it be published?
	What licenses do you need for your data?
	How should your data be cited when reused?
	Will there be an embargo period for (all or some of) the data?
	Are there other agreements or restrictions (see above) that need to be considered?
	Are there any legal/ethical restrictions that prevents the publication of all the material?
	Will these restrictions mean that action must be taken before the material can be
	made available?
	Is there a risk of delayed publication/making data available (all or parts of)?
	If so what might be needed to do to avoid this?
	
	\section{Discovering data}
	\subsection{Identification of needs}
	Do you plan to use existing data for your research?
	What is the purpose for which you need the data?
	What do you want to learn from the data?
	What type of data do you need?
	
	\subsection{Search for data}
	Do you know where the data may be located?
	How do you plan to search for the data?
	Evaluation of data quality
	What is the minimal required quality of the data (in terms of origin, contents, scope,
	size, methods, etc.)?
	How do you plan to evaluate data quality (evaluation of metadata, tests, analysis,
	comparisons)?
	
	\subsection{Gaining access to data}
	What are the (expected) terms and conditions for data access and use?
	What is the (expected) process for gaining access to the data?
	What is the (expected) time-span of the process for gaining access to the data?
	What are the (expected) costs for data access and use?
	
	
\end{document}
